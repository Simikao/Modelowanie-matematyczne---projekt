\documentclass[a4paper,11pt]{article}

\usepackage{amsfonts}
\usepackage{amsmath}
\usepackage{amssymb}
\usepackage{graphicx}
\usepackage[utf8]{inputenc}
\usepackage[polish]{babel}
\usepackage[T1]{fontenc}
\usepackage[margin=.9in, footskip=.25in, bottom=0.5in]{geometry}

\setlength{\abovecaptionskip}{-5pt plus 5pt minus 0pt}  

%----------------------
\def\\{\hfill\break}


%----------------------
\title{Projekt Egzaminacyjny}
\author{Wierzejski Szymon, Parzych Hubert}
\date{Listopad 2024 -- Styczeń 2025}

\begin{document}

\maketitle
\newpage

\tableofcontents
\newpage

\vspace*{4cm}
\section{Wstęp}

\textbf{Newag (NWG)} to polskie przedsiębiorstwo z siedzibą w Nowym Sączu, specjalizujące się w produkcji, modernizacji oraz serwisowaniu pojazdów szynowych. Firma powstała w 1995 roku, kontynuując tradycje zakładów naprawczych działających w regionie od połowy XX wieku. Oferta Newagu obejmuje m.in. lokomotywy elektryczne i spalinowe, zespoły trakcyjne oraz tramwaje. Jednym z flagowych produktów spółki jest nowoczesny elektryczny zespół trakcyjny Impuls, wykorzystywany przez wielu przewoźników w Polsce. Spółka zadebiutowała na Giełdzie Papierów Wartościowych w Warszawie w 2013 roku, odgrywając ważną rolę w polskim sektorze transportu szynowego.

\smallskip

\textbf{Dynatrace (DT.US)} to amerykańska firma technologiczna z siedzibą w Waltham, Massachusetts, zajmująca się dostarczaniem rozwiązań do monitorowania aplikacji, infrastruktur chmurowych oraz zarządzania wydajnością IT. Założona w 2005 roku firma specjalizuje się w wykorzystaniu sztucznej inteligencji i automatyzacji, pomagając przedsiębiorstwom w optymalizacji działania ich systemów IT. Platforma Dynatrace wspiera analizę danych w czasie rzeczywistym, umożliwiając szybkie rozwiązywanie problemów oraz poprawę jakości usług. Spółka zadebiutowała na Nasdaq Stock Market w 2019 roku i od tego czasu dynamicznie rozwija swoją działalność, współpracując z globalnymi liderami różnych branż.

\newpage

\section {Analiza cen spółek}
W ramach projektu egzaminacyjnego z modelowania matematycznego przeprowadzimy analizę cen akcji spółek Newag (NWG) oraz Dynatrace (DT.US). Analiza ta obejmie kolejno: statystyki opisowe, wykresy diagnostyczne oraz testowanie hipotezy o równości rozkładów cenowych obu spółek.

\subsection{Spółka NWG}
Rozpoczniemy od analizy cen akcji spółki NWG. Pierwszym krokiem będzie przyjrzenie się statystykom opisowym.

\subsubsection{Wykres cen zamknięcia akcji i histogram}
Poniżej znajdują się wykres kursów zamknięcia pokazujący zmiany w czasie oraz histogram gęstości dla cen spółki NWG

\begin{figure}[h]
\centering
\begin{minipage}[b]{0.40\textwidth}
\centering
\includegraphics[width=\textwidth, height=5.2cm, width = 7cm]{Wykres_cen_akcji_nwg.png}
\caption{Wykres cen zamknięcia akcji pokazujący zmiany w czasie - nwg}
\label{fig:r3}
\end{minipage}
\hfill
\begin{minipage}[b]{0.40\textwidth}
\centering
\includegraphics[width=\textwidth, height=5.25cm, width = 7cm]{wykres_zwrotów_nwg.png}
\caption{Wykres log-zwrotów cen zamknięcia - nwg}
\label{fig:r4}
\end{minipage}
\end{figure}


\subsubsection{Estymacja parametrów rozkładu log-zwrotów}

Zakładamy, że log-zwroty \(r_1, r_2, \ldots, r_n\) są niezależnymi realizacjami zmiennej losowej \(X\), która ma wartość oczekiwaną \(\mu\), wariancję \(\sigma^2\) oraz dystrybuantę \(F\). W tym rozdziale przeprowadzamy estymację parametrów \(\mu\) i \(\sigma^2\) oraz analizę kwantyli rozkładu na podstawie danych log-zwrotów.

\smallskip
\textbf{Wartość oczekiwana}

Wartość oczekiwana log-zwrotów, oznaczona jako \(\mu\), reprezentuje średnią zmianę cen akcji w badanym okresie. Estymator wartości oczekiwanej \(\hat{\mu}\) wyraża się wzorem:
\[
\hat{\mu} = \frac{1}{n} \sum_{i=1}^{n} r_i,
\]
gdzie \(n\) to liczba obserwacji, a \(r_i\) to log-zwrot w dniu \(i\). Wyestymowana wartość oczekiwana wynosi:  
\[
\hat{\mu} = 4.74849 \times 10^{-5}.
\]

\textbf{Wariancja i odchylenie standardowe}

Wariancja \(\sigma^2\) mierzy, jak bardzo log-zwroty odchylają się od wartości oczekiwanej. Estymator wariancji \(\hat{\sigma}^2\) jest dany wzorem:
\[
\hat{\sigma}^2 = \frac{1}{n-1} \sum_{i=1}^{n} (r_i - \hat{\mu})^2.
\]
Odchylenie standardowe, oznaczone jako \(\hat{\sigma}\), to pierwiastek kwadratowy z estymatora wariancji:
\[
\hat{\sigma} = \sqrt{\hat{\sigma}^2}.
\]
Wyestymowane wartości wynoszą:  
\[
\hat{\sigma}^2 = 0.000463, \quad \hat{\sigma} = 0.0215.
\]

\textbf{Kwantyle}

Kwantyle wyznaczają wartości log-zwrotów, poniżej których znajduje się określony procent obserwacji. Kwantyle rzędów \(\alpha = 5\%, 50\%, 95\%\) estymujemy na podstawie dystrybuanty empirycznej \(F_n\):
\[
q(\alpha) = \hat{Q}(\alpha),
\]
gdzie \(\hat{Q}\) to klasyczny estymator kwantyli. Wyestymowane wartości wynoszą:  
\[
q(5\%) = -0.0312, \quad q(50\%) = 0.0000, \quad q(95\%) = 0.0360.
\]

\textbf{Wyniki estymacji}

Wyniki przedstawiono w poniższej tabeli:
\[
\begin{array}{|c|c|c|c|c|c|}
\hline
\hat{\mu} & \hat{\sigma}^2 & \hat{\sigma} & q(5\%) & q(50\%) & q(95\%) \\
\hline
4.74849 \times 10^{-5} & 0.000463 & 0.0215 & -0.0312 & 0.0000 & 0.0360 \\
\hline
\end{array}
\]

\textbf{Interpretacja kwantyli}

\begin{itemize}
    \item \textbf{Kwantyl 5\%} (\(q(5\%) = -0.0312\)) oznacza, że 5\% log-zwrotów było mniejszych niż \(-3.12\%\).
    \item \textbf{Mediana (kwantyl 50\%)} (\(q(50\%) = 0.0000\)) wskazuje, że połowa log-zwrotów jest równa lub mniejsza od \(0\).
    \item \textbf{Kwantyl 95\%} (\(q(95\%) = 0.0360\)) oznacza, że 95\% log-zwrotów było mniejszych niż \(3.60\%\).
\end{itemize}

Wartości te zostały również zaprezentowane na histogramie log-zwrotów, na którym zaznaczono średnią oraz oszacowane kwantyle.

\begin{figure}[h]
\centering
\includegraphics{estymacja_kwantyli_nwg.png}
\caption{Histogram log-zwrotów z naniesionymi kwantylami, medianą i średnią dla NWG}
\end{figure}

\subsubsection{Estymator dystrybuanty empirycznej}

Dystrybuanta empiryczna \( F_n(x) \) jest estymatorem dystrybuanty \( F(x) \) zmiennej losowej \( X \), który definiuje się wzorem:

\[
F_n(x) = \frac{1}{n} \sum_{i=1}^n \mathbf{1}(r_i \leq x),
\]

gdzie:
\begin{itemize}
    \item \( n \) to liczba obserwacji,
    \item \( r_i \) to log-zwrot w dniu \( i \),
    \item \( \mathbf{1}(r_i \leq x) \) to funkcja wskaźnikowa, która przyjmuje wartość 1, jeśli \( r_i \leq x \), i 0 w przeciwnym razie.
\end{itemize}

Funkcja \( F_n(x) \) wyznacza proporcję log-zwrotów mniejszych lub równych \( x \). Wartości \( F_n(x) \) rosną monotonicznie od 0 do 1 w miarę zwiększania \( x \).

\begin{figure}[h]
\centering
\includegraphics{dystrybuanta_empiryczna_nwg.png}
\caption{Wykres wyestymowanej dystrybuanty empirycznej do spółki NWG}
\end{figure}





\subsubsection{Wykresy diagnostyczne. Analiza wartości statystyk.}
Poniżej przedstawiony jest zbiór wybranych wykresów diagnostycznych; histogram z rokładami teoretycznymi, wykres kwantylowy, dystrybuanta i wykres P-P
\begin{figure}[h]
\centering
\includegraphics[width=12cm, height=12cm]{wykresy_diagnostyczne_nwg.png}
\caption{Wykresy diagnostyczne - NWG}
\end{figure}

\textbf{Rozkłady prawdopodobieństwa} naniesione na histogram gęstości danych. Wykres przedstawia trzy rozkłady teoretyczne (normalny, log-normalny, gamma) dopasowane parametrami do histogramu gęstości danych. 

\textbf{Wykres kwantylowy QQ-plot} to graficzne narzędzie, które porównuje rozkłady prawdopodobieństwa przez wykreślenie ich kwantyli na przeciwnych osiach (x,y). W przypadku, gdy dane są zgodne z danym rozkładem teoretycznym, punkty na wykresie układają się wzdłuż prostej \(y = x\).

\textbf{CDF}, czyli \textbf{Cumulative Distribution Function} (Dystrybuanta), to funkcja rzeczywista wyznaczająca rozkład prawdopodobieństwa. Przyporządkowuje każdej możliwej wartości zmiennej losowej pewne prawdopodobieństwo, że ta wartość jest mniejsza lub równa danej liczbie. Maksymalnie dystrybuanta wynosi 1 a minimalnie 0. CDF określa, jak prawdopodobieństwo rozkłada się względem konkretnych wartości zmiennych losowych.

\textbf{Wykres PP-plot} to narzędzie, które działa na podobnej zasadzie co QQ-plot, z tą różnicą, że zamiast kwantyli porónywane są prawdopodobieństwa kumulacyjne, czyli dystrybuanty; empiryczna i teoretyczna. W przypadku, gdy dane są zgodne z danym rozkładem teoretycznym, punkty na wykresie układają się wzdłuż prostej \(y = x\).

Wykresy diagnostyczne w kontekście statystyki i analizy danych służą do oceny adekwatności dopasowania modelu do danych i do znalezienia potencjalnych problemów określonego rozkładu.

Na powyższych wykresach widać, że wykresy przedstawiają się zupełnie inaczej. Dla wszysktich wykresów jest widoczne, że najbardziej dystrybucja normalna najlepiej pokrywa nasz wykres. W celu jednoznacznego określenia i potwierdzenia wyników wykorzystam analizę statystyk KS, CM i AD oraz kryteriów informacyjnych AIC i BIC

\vspace*{0.5cm}

\textbf{Analiza matematyczna}

Statystyki te służą do sprawdzania, czy dane zgadzają się z rozkładami: normalnym, log-normalnym i gamma, poprzez analizę różnic między empiryczną dystrybuantą danych a dystrybuantą rozkładów teoretycznych.

Za pomocą tych statystyk możemy precyzyjniej określić, który rozkład najlepiej odpowiada danym, porównując je dla różnych rozkładów w ramach każdej statystyki. Wybieramy ten rozkład, który charakteryzuje się najmniejszą wartością statystyki testowej w większości przypadków, co oznacza, że jego dystrybuanta jest najbardziej zbliżona do empirycznej dystrybuanty danych.

Test \textbf{Kołmogorowa-Smirnowa} jest oparty o największą odległość pomiędzy dystrybuantami empiryczną \(F_n\) i teoretyczną \(F\)

Statystyka \textbf{Craméra-von Misesa} pozwala na ocenę, na ile dane empiryczne zgadzają się z przyjętym teoretycznym rozkładem na przestrzeni całego rozkładu, a nie jak w przypadku KS - punktowej wartości maksymalnej

Test \textbf{Andersona-Darlinga} jest oparty o ważoną odległość Cramera von Misesa pomiędzy dystrybuantami empiryczną \(F_n\) i teoretyczną \(F\) z wagami odpowiadającymi odwrotności wariancji dystrybuanty empirycznej

\vspace*{0.5cm}
\centerline{Statystyki:}
\vspace*{0.2cm}
\centerline{\begin{tabular}{|l|c|c|c|}
\hline
& r. normalny & t-studenta  \\
\hline
KS & 0.08888125 & 0.4717643  \\
\hline
CM & 0.98860358 & 39.4555480  \\
\hline
AD & 5.60134046 & 184.2992521  \\
\hline
\end{tabular}}

\vspace*{0.5cm}
\centerline{Kryteria informacyjne:}
\vspace*{0.2cm}

\centerline{\begin{tabular}{|l|c|c|c|}
\hline
& r. normalny & t-studenta  \\
\hline
AIC & -2416.931 & 921.9655  \\
\hline
BIC & -2408.502 & 926.1802  \\
\hline
\end{tabular}}

\vspace*{1cm}

\newpage
Analizując podane statystyki i kryteria informacyjne, wyniki wskazują, że najniższe wartości statystyk są dla rozkładu normalnego, czyli jest on najlepszym modelem opisującym kursy zamknięcia. W każdym kryterium informacyjnym, rozkład normalny również osiąga najniższe wartości. Oznacza to, że zdecydowanie ma on najlepsze dopasowanie do danych kursów zamknięcia.

\subsubsection{Hipoteza o równości rozkładów}
Następnym przedmiotem analizy dotyczącym spółki Newag jest badanie hipotezy o równości rozkładów. Test Kołmogorowa-Smirnowa (KS) jest używany do porównania dwóch rozkładów prawdopodobieństwa i określenia, jaka jest maksymalna odległość między tymi rozkładami na podanych danych. Przedsawione poniżej są wyniki tego testu wraz z histogramem gęstości wyników testu KS przeprowadzonego 10000 razy na próbkach 100 elementowych generowanych z rozkładu log-normalnego opisującego według hipotezy zerowej ceny spółki NWG.
\begin{figure}[h]
\centering
\includegraphics[width=9cm, height=12cm]{hipoteza_o_rownosci_nwg.png}
\caption{Histogram gęstości wyników testu KS próbek metody Monte-Carlo - NWG}
\end{figure}

Hipoteza zerowa \(H_0\) zakłada, że za pomocą rozkładu normalnego można opisywać ceny spółki NWG. Hipoteza alternatywna \(H_1\) ma przeciwne założenia: za pomocą rozkładu normalnego nie można opisywać cen.

Metoda Monte-Carlo, którą testujemy hipotezę zerową polega na wielokrotnym (w tym przypadku 10000-krotnym) generowaniu próbek n-elementowych (tutaj: n=100) z określonego rozkładu, a następnie poddaniu każdorazowo próbki określonemu testowi (KS). 

P-value, czyli procent danych otrzymanych metodą MC, które są większe od testu KS dla oryginalnych danych wynosi praktycznie 0\%. Test sugeruje, że na danym poziomie istotności, wynoszącym 5\% są podstawy do odrzucenia hipotezy zerowej, zakładającej, że badane dane mają rozkład normalny.









\subsection{Spółka DT.US}
Kontynuując analizę cen akcji, przechodzimy do spółki Dynatrace. Pierwszym krokiem będzie przyjrzenie się statystykom opisowym.

\subsubsection{Wykres cen zamknięcia akcji i histogram}
Poniżej przedstawiamy wykres kursów zamknięcia, który ilustruje zmiany w czasie, oraz histogram gęstości dla cen akcji spółki Dynatrace.

\begin{figure}[h]
\centering
\begin{minipage}[b]{0.40\textwidth}
\centering
\includegraphics[width=\textwidth, height=5.2cm, width = 7cm]{Wykres_cen_akcji_dt.png}
\caption{Wykres cen zamknięcia akcji pokazujący zmiany w czasie DT.US}
\label{fig:r3}
\end{minipage}
\hfill
\begin{minipage}[b]{0.40\textwidth}
\centering
\includegraphics[width=\textwidth, height=5.25cm, width = 7cm]{wykres_zwrotów_nwg.png}
\caption{Wykres log-zwrotów cen zamknięcia DT.US}
\label{fig:r4}
\end{minipage}
\end{figure}


\subsubsection{Estymacja parametrów rozkładu log-zwrotów}

Zakładamy, że log-zwroty \(r_1, r_2, \ldots, r_n\) są niezależnymi realizacjami zmiennej losowej \(X\), która ma wartość oczekiwaną \(\mu\), wariancję \(\sigma^2\) oraz dystrybuantę \(F\). W tym rozdziale przeprowadzamy estymację parametrów \(\mu\) i \(\sigma^2\) oraz analizę kwantyli rozkładu na podstawie danych log-zwrotów.

\smallskip
\textbf{Wartość oczekiwana}

Wartość oczekiwana log-zwrotów, oznaczona jako \(\mu\), reprezentuje średnią zmianę cen akcji w badanym okresie. Estymator wartości oczekiwanej \(\hat{\mu}\) wyraża się wzorem:
\[
\hat{\mu} = \frac{1}{n} \sum_{i=1}^{n} r_i,
\]
gdzie \(n\) to liczba obserwacji, a \(r_i\) to log-zwrot w dniu \(i\). Wyestymowana wartość oczekiwana wynosi:  
\[
\hat{\mu} = 1.618569 \times 10^{-4}.
\]

\textbf{Wariancja i odchylenie standardowe}

Wariancja \(\sigma^2\) mierzy, jak bardzo log-zwroty odchylają się od wartości oczekiwanej. Estymator wariancji \(\hat{\sigma}^2\) jest dany wzorem:
\[
\hat{\sigma}^2 = \frac{1}{n-1} \sum_{i=1}^{n} (r_i - \hat{\mu})^2.
\]
Odchylenie standardowe, oznaczone jako \(\hat{\sigma}\), to pierwiastek kwadratowy z estymatora wariancji:
\[
\hat{\sigma} = \sqrt{\hat{\sigma}^2}.
\]
Wyestymowane wartości wynoszą:  
\[
\hat{\sigma}^2 = 0.0009134114, \quad \hat{\sigma} = 0.0302227.
\]

\textbf{Kwantyle}

Kwantyle wyznaczają wartości log-zwrotów, poniżej których znajduje się określony procent obserwacji. Kwantyle rzędów \(\alpha = 5\%, 50\%, 95\%\) estymujemy na podstawie dystrybuanty empirycznej \(F_n\):
\[
q(\alpha) = \hat{Q}(\alpha),
\]
gdzie \(\hat{Q}\) to klasyczny estymator kwantyli. Wyestymowane wartości wynoszą:  
\[
q(5\%) = -0.0480, \quad q(50\%) = 0.0010, \quad q(95\%) = 0.0439.
\]

\textbf{Wyniki estymacji}

Wyniki przedstawiono w poniższej tabeli:
\[
\begin{array}{|c|c|c|c|c|c|}
\hline
\hat{\mu} & \hat{\sigma}^2 & \hat{\sigma} & q(5\%) & q(50\%) & q(95\%) \\
\hline
\hat{\mu} = 1.618569 \times 10^{-4} & 0.0009134114 & 0.0302227 & -0.0480 & 0.0010 & 0.0439 \\
\hline
\end{array}
\]

\textbf{Interpretacja kwantyli}

\begin{itemize}
    \item \textbf{Kwantyl 5\%} (\(q(5\%) = -0.0480\)) oznacza, że 5\% log-zwrotów było mniejszych niż \(-4.80\%\).
    \item \textbf{Mediana (kwantyl 50\%)} (\(q(50\%) = 0.0010\)) wskazuje, że połowa log-zwrotów jest równa lub mniejsza od \(0.10\%\).
    \item \textbf{Kwantyl 95\%} (\(q(95\%) = 0.0439\)) oznacza, że 95\% log-zwrotów było mniejszych niż \(4.39\%\).
\end{itemize}

Wartości te zostały również zaprezentowane na histogramie log-zwrotów, na którym zaznaczono średnią oraz oszacowane kwantyle.

\begin{figure}[h]
\centering
\includegraphics{estymacja_kwantyli_nwg.png}
\caption{Histogram log-zwrotów z naniesionymi kwantylami, medianą i średnią dla NWG}
\end{figure}

\subsubsection{Estymator dystrybuanty empirycznej}

Dystrybuanta empiryczna \( F_n(x) \) jest estymatorem dystrybuanty \( F(x) \) zmiennej losowej \( X \), który definiuje się wzorem:

\[
F_n(x) = \frac{1}{n} \sum_{i=1}^n \mathbf{1}(r_i \leq x),
\]

gdzie:
\begin{itemize}
    \item \( n \) to liczba obserwacji,
    \item \( r_i \) to log-zwrot w dniu \( i \),
    \item \( \mathbf{1}(r_i \leq x) \) to funkcja wskaźnikowa, która przyjmuje wartość 1, jeśli \( r_i \leq x \), i 0 w przeciwnym razie.
\end{itemize}

Funkcja \( F_n(x) \) wyznacza proporcję log-zwrotów mniejszych lub równych \( x \). Wartości \( F_n(x) \) rosną monotonicznie od 0 do 1 w miarę zwiększania \( x \).

\begin{figure}[h]
\centering
\includegraphics{dystrybuanta_empiryczna_dt.png}
\caption{Wykres wyestymowanej dystrybuanty empirycznej spółki DT.US}
\end{figure}





\subsubsection{Wykresy diagnostyczne. Analiza wartości statystyk.}
Poniżej przedstawiony jest zbiór wybranych wykresów diagnostycznych; histogram z rokładami teoretycznymi, wykres kwantylowy, dystrybuanta i wykres P-P
\begin{figure}[h]
\centering
\includegraphics[width=12cm, height=12cm]{wykresy_diagnostyczne_dt.png}
\caption{Wykresy diagnostyczne -- DT.US}
\end{figure}

\textbf{Rozkłady prawdopodobieństwa} naniesione na histogram gęstości danych. Wykres przedstawia trzy rozkłady teoretyczne (normalny, log-normalny, gamma) dopasowane parametrami do histogramu gęstości danych. 

\textbf{Wykres kwantylowy QQ-plot} to graficzne narzędzie, które porównuje rozkłady prawdopodobieństwa przez wykreślenie ich kwantyli na przeciwnych osiach (x,y). W przypadku, gdy dane są zgodne z danym rozkładem teoretycznym, punkty na wykresie układają się wzdłuż prostej \(y = x\).

\textbf{CDF}, czyli \textbf{Cumulative Distribution Function} (Dystrybuanta), to funkcja rzeczywista wyznaczająca rozkład prawdopodobieństwa. Przyporządkowuje każdej możliwej wartości zmiennej losowej pewne prawdopodobieństwo, że ta wartość jest mniejsza lub równa danej liczbie. Maksymalnie dystrybuanta wynosi 1 a minimalnie 0. CDF określa, jak prawdopodobieństwo rozkłada się względem konkretnych wartości zmiennych losowych.

\textbf{Wykres PP-plot} to narzędzie, które działa na podobnej zasadzie co QQ-plot, z tą różnicą, że zamiast kwantyli porónywane są prawdopodobieństwa kumulacyjne, czyli dystrybuanty; empiryczna i teoretyczna. W przypadku, gdy dane są zgodne z danym rozkładem teoretycznym, punkty na wykresie układają się wzdłuż prostej \(y = x\).

Wykresy diagnostyczne w kontekście statystyki i analizy danych służą do oceny adekwatności dopasowania modelu do danych i do znalezienia potencjalnych problemów określonego rozkładu.

Na powyższych wykresach widać, że wykresy przedstawiają się zupełnie inaczej. Dla wszysktich wykresów jest widoczne, że najbardziej dystrybucja normalna najlepiej pokrywa nasz wykres. W celu jednoznacznego określenia i potwierdzenia wyników wykorzystam analizę statystyk KS, CM i AD oraz kryteriów informacyjnych AIC i BIC

\vspace*{0.5cm}

\textbf{Analiza matematyczna}

Statystyki te służą do sprawdzania, czy dane zgadzają się z rozkładami: normalnym, log-normalnym i gamma, poprzez analizę różnic między empiryczną dystrybuantą danych a dystrybuantą rozkładów teoretycznych.

Za pomocą tych statystyk możemy precyzyjniej określić, który rozkład najlepiej odpowiada danym, porównując je dla różnych rozkładów w ramach każdej statystyki. Wybieramy ten rozkład, który charakteryzuje się najmniejszą wartością statystyki testowej w większości przypadków, co oznacza, że jego dystrybuanta jest najbardziej zbliżona do empirycznej dystrybuanty danych.

Test \textbf{Kołmogorowa-Smirnowa} jest oparty o największą odległość pomiędzy dystrybuantami empiryczną \(F_n\) i teoretyczną \(F\)

Statystyka \textbf{Craméra-von Misesa} pozwala na ocenę, na ile dane empiryczne zgadzają się z przyjętym teoretycznym rozkładem na przestrzeni całego rozkładu, a nie jak w przypadku KS - punktowej wartości maksymalnej

Test \textbf{Andersona-Darlinga} jest oparty o ważoną odległość Cramera von Misesa pomiędzy dystrybuantami empiryczną \(F_n\) i teoretyczną \(F\) z wagami odpowiadającymi odwrotności wariancji dystrybuanty empirycznej

\vspace*{0.5cm}
\centerline{Statystyki:}
\vspace*{0.2cm}
\centerline{\begin{tabular}{|l|c|c|}
\hline
& r. normalny & t-studenta  \\
\hline
KS & 0.0953095 & 0.4630927  \\
\hline
CM & 1.1377620 & 38.6249487  \\
\hline
AD & 6.1072555 & 180.9702115  \\
\hline
\end{tabular}}

\vspace*{0.5cm}
\centerline{Kryteria informacyjne:}
\vspace*{0.2cm}

\centerline{\begin{tabular}{|l|c|c|}
\hline
& r. normalny & t-studenta  \\
\hline
AIC & -2077.225 & 922.1910  \\
\hline
BIC & -2068.795 & 926.4056  \\
\hline
\end{tabular}}

\vspace*{1cm}

\newpage
Analizując podane statystyki i kryteria informacyjne, wyniki wskazują, że rozkład normalny ma lepsze dopasowanie do danych kursów zamknięcia w porównaniu do rozkładu t-Studenta. Statystyki Kolmogorowa-Smirnowa, Cramera-von Misesa oraz Andersona-Darlinga dla rozkładu normalnego są znacznie niższe niż dla rozkładu t-Studenta, co sugeruje, że rozkład normalny lepiej opisuje te dane.

Dodatkowo, w każdym z kryteriów informacyjnych (Akaike's Information Criterion oraz Bayesian Information Criterion) rozkład normalny osiąga najniższe wartości, co potwierdza jego przewagę jako modelu opisującego kursy zamknięcia. W związku z tym, można stwierdzić, że rozkład normalny jest najlepszym modelem dla analizowanych danych.

\subsubsection{Hipoteza o równości rozkładów}
Kolejnym aspektem analizy dotyczącej spółki DT.US jest ocena hipotezy o równości rozkładów. W tym celu zastosowano test Kołmogorowa-Smirnowa (KS), który służy do porównania dwóch rozkładów prawdopodobieństwa oraz do określenia maksymalnej różnicy między nimi na podstawie dostępnych danych. Poniżej przedstawiono wyniki tego testu, a także histogram gęstości, który ilustruje wyniki testu KS przeprowadzonego 10000 razy na próbkach składających się z 100 elementów, generowanych z rozkładu log-normalnego, który według hipotezy zerowej ma na celu opisanie cen spółki DT.US.
\begin{figure}[h]
\centering
\includegraphics[width=9cm, height=12cm]{hipoteza_o_rownosci_dt.png}
\caption{Histogram gęstości wyników testu KS próbek metody Monte-Carlo -- DT.US}
\end{figure}

Hipoteza zerowa \(H_0\) zakłada, że za pomocą rozkładu normalnego można opisywać ceny spółki NWG. Hipoteza alternatywna \(H_1\) ma przeciwne założenia: za pomocą rozkładu normalnego nie można opisywać cen.

Metoda Monte-Carlo, którą testujemy hipotezę zerową polega na wielokrotnym (w tym przypadku 10000-krotnym) generowaniu próbek n-elementowych (tutaj: n=100) z określonego rozkładu, a następnie poddaniu każdorazowo próbki określonemu testowi (KS). 

P-value, czyli procent danych otrzymanych metodą MC, które są większe od testu KS dla oryginalnych danych wynosi praktycznie 0\%. Test sugeruje, że na danym poziomie istotności, wynoszącym 5\% są podstawy do odrzucenia hipotezy zerowej, zakładającej, że badane dane mają rozkład normalny.



\section{Analiza łącznego rozkładu log-zwrotów}

W tej części przeprowadzimy analizę wektora log-zwrotów dwóch akcji: Newag (NWG) oraz Dynatrace (DT.US). 

\subsection{Założenia i estymacja parametrów}
Załóżmy, że log-zwroty dwóch akcji są niezależnymi realizacjami wektora losowego \((X, Y)\) o nieznanej gęstości \(f\), wektorze średnich \((\mu_1, \mu_2)\), współczynniku korelacji \(\rho\), macierzy kowariancji \(\Sigma\) i macierzy korelacji \(P\). Estymatory tych parametrów wyrażają się wzorami:

\[
\hat{\mu}_1 = \frac{1}{n} \sum_{i=1}^{n} X_i, \quad \hat{\mu}_2 = \frac{1}{n} \sum_{i=1}^{n} Y_i
\]

\[
\hat{\sigma}_{X}^2 = \frac{1}{n-1} \sum_{i=1}^{n} (X_i - \hat{\mu}_1)^2, \quad \hat{\sigma}_{Y}^2 = \frac{1}{n-1} \sum_{i=1}^{n} (Y_i - \hat{\mu}_2)^2
\]

\[
\hat{\sigma}_{XY} = \frac{1}{n-1} \sum_{i=1}^{n} (X_i - \hat{\mu}_1)(Y_i - \hat{\mu}_2)
\]

\[
\hat{\rho} = \frac{\hat{\sigma}_{XY}}{\hat{\sigma}_X \hat{\sigma}_Y}
\]

\[
\hat{\Sigma} = \begin{pmatrix}
\hat{\sigma}_X^2 & \hat{\sigma}_{XY} \\
\hat{\sigma}_{XY} & \hat{\sigma}_Y^2
\end{pmatrix}, \quad \hat{P} = \begin{pmatrix}
1 & \hat{\rho} \\
\hat{\rho} & 1
\end{pmatrix}
\]

\subsection{Wykres rozrzutu z histogramami rozkładów brzegowych}
Poniżej przedstawiamy wykres rozrzutu log-zwrotów akcji NWG i DT.US wraz z histogramami rozkładów brzegowych.

\begin{figure}[h]
\centering
\includegraphics{rozrzut_z_histogramami.png}
\caption{Wykres rozrzutu log-zwrotów z histogramami rozkładów brzegowych}
\end{figure}

Wykres rozrzutu przedstawia zależność między log-zwrotami akcji DT.US (oś X) i NWG (oś Y). Każdy punkt na wykresie reprezentuje parę log-zwrotów dla obu spółek w danym dniu. Histogramy po bokach wykresu pokazują rozkłady brzegowe log-zwrotów dla każdej z akcji osobno.

Większość punktów skupia się wokół środka wykresu, co sugeruje, że log-zwroty obu akcji mają tendencję do oscylowania wokół swoich średnich wartości. Histogramy brzegowe wskazują, że rozkłady log-zwrotów dla obu akcji są zbliżone do rozkładu normalnego, co jest zgodne z wcześniejszymi analizami.

Wykres rozrzutu pozwala również na ocenę korelacji między log-zwrotami obu akcji. Jeśli punkty na wykresie układają się wzdłuż linii prostej, oznacza to silną korelację między zmiennymi. W przypadku braku wyraźnego wzorca lub gdy punkty są rozproszone, korelacja jest słaba lub nieistniejąca. Na przedstawionym wykresie punkty są dość rozproszone, co sugeruje, że nie ma wyraźnej korelacji między log-zwrotami akcji DT.US i NWG.

Rozkłady brzegowe są symetryczne, co dodatkowo potwierdza, że log-zwroty obu akcji mogą być dobrze modelowane za pomocą rozkładu normalnego.

Wykres rozrzutu z histogramami rozkładów brzegowych dostarcza cennych informacji na temat struktury danych i potwierdza, że rozkład normalny jest odpowiednim modelem dla log-zwrotów obu akcji.


\subsection{Gęstość rozkładu normalnego i rozkładów brzegowych}
Gęstość łączna rozkładu normalnego o wyestymowanych parametrach \(N(\hat{\mu}, \hat{\Sigma})\) wyraża się wzorem:

\[
\hat{f}(x, y) = \frac{1}{2\pi \sqrt{|\hat{\Sigma}|}} \exp \left( -\frac{1}{2} (z^T \hat{\Sigma}^{-1} z) \right)
\]

gdzie \(z = \begin{pmatrix}
x - \hat{\mu}_1 \\
y - \hat{\mu}_2
\end{pmatrix}\).

Gęstości brzegowe mają postać:

\[
\hat{f}_X(x) = \frac{1}{\sqrt{2\pi \hat{\sigma}_X^2}} \exp \left( -\frac{(x - \hat{\mu}_1)^2}{2\hat{\sigma}_X^2} \right)
\]

\[
\hat{f}_Y(y) = \frac{1}{\sqrt{2\pi \hat{\sigma}_Y^2}} \exp \left( -\frac{(y - \hat{\mu}_2)^2}{2\hat{\sigma}_Y^2} \right)
\]

Przedstawiamy wykres gęstości łącznej oraz wykresy gęstości brzegowych.

\begin{figure}[h]
\centering
\includegraphics{zwroty_gestosc_laczona.png}
\caption{Wykres gęstości łącznej log-zwrotów NWG i DT.US}
\end{figure}

\begin{figure}[h]
\centering
\includegraphics{zwroty_wykresy_jednowymiarowe.png}
\caption{Wykresy gęstości brzegowych log-zwrotów NWG i DT.US}
\end{figure}

Wykres gęstości łącznej przedstawia trójwymiarową powierzchnię, która ilustruje, jak prawdopodobieństwo rozkłada się w przestrzeni log-zwrotów obu akcji. Najwyższe punkty na wykresie odpowiadają największemu prawdopodobieństwu wystąpienia określonych par log-zwrotów. Wykres ten pozwala na wizualizację zależności między log-zwrotami akcji NWG i DT.US.

Wykresy gęstości brzegowych przedstawiają jednowymiarowe rozkłady prawdopodobieństwa log-zwrotów dla każdej z akcji osobno. Wykresy te pokazują, jak prawdopodobieństwo rozkłada się wzdłuż osi log-zwrotów dla akcji NWG i DT.US. Oba wykresy gęstości brzegowych są symetryczne i mają kształt zbliżony do rozkładu normalnego, co potwierdza wcześniejsze analizy.

Analizując wykresy, można zauważyć, że gęstość łączna jest skoncentrowana wokół środka, co sugeruje, że log-zwroty obu akcji mają tendencję do oscylowania wokół swoich średnich wartości. Wykresy gęstości brzegowych dodatkowo potwierdzają, że log-zwroty obu akcji mogą być dobrze modelowane za pomocą rozkładu normalnego.

\subsection{Generowanie próby z rozkładu normalnego}
Wygenerowaliśmy próbę o liczności równej liczbie danych z rozkładu \(N(\hat{\mu}, \hat{\Sigma})\). Poniżej przedstawiamy wykres rozrzutu wygenerowanej próby oraz porównanie z wykresem rozrzutu danych.

\begin{figure}[h]
\centering
\includegraphics{diff_gestosc_laczona_detailed.png}
\caption{Porównanie wykresu rozrzutu danych i wygenerowanej próby}
\end{figure}

Wykres rozrzutu wygenerowanej próby przedstawia zależność między log-zwrotami akcji DT.US (oś X) i NWG (oś Y) dla danych wygenerowanych z rozkładu normalnego o wyestymowanych parametrach. Porównując ten wykres z wykresem rozrzutu rzeczywistych danych, można ocenić, jak dobrze rozkład normalny opisuje rzeczywiste dane.

Na obu wykresach punkty są skoncentrowane wokół środka, co sugeruje, że zarówno rzeczywiste dane, jak i wygenerowana próba mają podobne średnie wartości log-zwrotów. Rozproszenie punktów na obu wykresach jest również podobne, co wskazuje na podobną wariancję log-zwrotów w rzeczywistych danych i wygenerowanej próbie.

Porównanie wykresów rozrzutu danych i wygenerowanej próby potwierdza, że rozkład normalny z wyestymowanymi parametrami dobrze opisuje log-zwroty akcji NWG i DT.US.


\subsection{Ocena dobroci dopasowania rozkładu normalnego}

Analizując wykresy rozrzutu danych i wygenerowanej próby, można ocenić dobroć dopasowania rozkładu normalnego. Widać, że rozkład normalny z wyestymowanymi parametrami dobrze opisuje dane, co potwierdzają podobieństwa w strukturze rozrzutu.

Na obu wykresach punkty są skoncentrowane wokół środka, co sugeruje, że zarówno rzeczywiste dane, jak i wygenerowana próba mają podobne średnie wartości log-zwrotów. Rozproszenie punktów na obu wykresach jest również podobne, co wskazuje na podobną wariancję log-zwrotów w rzeczywistych danych i wygenerowanej próbie.

Dodatkowo, brak wyraźnych odchyleń od linii prostej na wykresach rozrzutu sugeruje, że nie ma istotnych różnic między rzeczywistymi danymi a danymi wygenerowanymi z rozkładu normalnego. To potwierdza, że rozkład normalny jest odpowiednim modelem dla log-zwrotów obu akcji.

Porównanie wykresów rozrzutu danych i wygenerowanej próby potwierdza, że rozkład normalny z wyestymowanymi parametrami dobrze opisuje log-zwroty akcji NWG i DT.US.


\section{Podsumowanie}
W ramach projektu egzaminacyjnego z modelowania matematycznego przeprowadziliśmy szczegółową analizę cen akcji dwóch spółek: polskiej spółki Newag (NWG) oraz amerykańskiej spółki technologicznej Dynatrace (DT.US). Analiza obejmowała statystyki opisowe, wykresy diagnostyczne oraz testowanie hipotezy o równości rozkładów cenowych obu spółek.

Dla każdej ze spółek przeprowadziliśmy estymację parametrów rozkładu log-zwrotów, w tym wartości oczekiwanej, wariancji oraz kwantyli. Następnie, za pomocą wykresów diagnostycznych, oceniliśmy dopasowanie różnych rozkładów teoretycznych do danych empirycznych. Wyniki wskazały, że rozkład normalny najlepiej opisuje log-zwroty obu spółek.

W dalszej części pracy przeanalizowaliśmy łączny rozkład log-zwrotów obu akcji, estymując parametry wektora średnich, współczynnika korelacji, macierzy kowariancji oraz macierzy korelacji. Wykres rozrzutu z histogramami rozkładów brzegowych pozwolił na ocenę zależności między log-zwrotami obu akcji, wskazując na brak wyraźnej korelacji między nimi.

Przeprowadziliśmy również generowanie próby z rozkładu normalnego o wyestymowanych parametrach i porównaliśmy wykres rozrzutu wygenerowanej próby z wykresem rozrzutu rzeczywistych danych. Analiza wykazała, że rozkład normalny z wyestymowanymi parametrami dobrze opisuje log-zwroty obu akcji.

Podsumowując, porównaliśmy dwie spółki działające w zupełnie różnych branżach i na różnych rynkach: Newag, polską firmę zajmującą się produkcją taboru kolejowego, oraz Dynatrace, amerykańską firmę technologiczną specjalizującą się w monitorowaniu aplikacji i infrastruktur chmurowych. Pomimo różnic w branżach i odbiorcach ich usług, analiza wykazała, że log-zwroty obu spółek mogą być dobrze modelowane za pomocą rozkładu normalnego. Wyniki te sugerują, że niezależnie od specyfiki branży, rozkład normalny może być użytecznym narzędziem do modelowania log-zwrotów cen akcji.




\end{document}

